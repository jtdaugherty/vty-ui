\section{Focus Groups and Focus Changes}
\label{sec:focus}

Graphical interfaces allow the user to change focus between all of the
primary interface input elements, usually with the Tab key.  The same is
true in vty-ui, except that because any widget can accept events -- and
because which widgets the focus is a policy decision that you get to
make -- the library cannot automatically determine which widgets should
get the focus, or the order in which focus should be received.  As a
result, vty-ui provides a type called a "focus group."

A focus group is just an ordered sequence of widgets that should get the
user's focus as the Tab key is pressed.  Widgets receive focus in the
order in which they are added to the group, and the first widget to be
added automatically gets the focus when the interface is set up.

Creating a focus group is simple:

\begin{minted}{haskell}
fg <- newFocusGroup
\end{minted}

Adding widgets to focus groups is also straightforward:

\begin{minted}{haskell}
w <- someWidget
addToFocusGroup fg w
\end{minted}

A widget's ``focused behavior'' depends entirely on the widget's
implementation.  Some widgets, when focused, provide a text cursor;
others merely change foreground and background color.  In any case,
the widgets that the user can interact with should be in the
interface's focus group.

Once the widgets are added to the focus group, you won't have to manage
anything else; the Tab key event is intercepted by the FocusGroup
itself, and user input events are passed to the focused widget until the
focus is changed.

If, for some reason, you would like to be notified when a widget
receives or loses focus, you may register event handlers for these
events on any widget:

\begin{minted}{haskell}
w <- someWidget
w `onGainFocus` \this -> ...
w `onLoseFocus` \this -> ...
\end{minted}

In both cases above, the \fw{this} parameter to each event handler is
just the widget to which the event handler is being attached (i.e.,
\fw{w}).  Many event handlers follow this pattern.

\subsection{Top-Level Key Event Handlers}

All user input is handled via a focus group; the focus state of the
group indicates which widget will receive user input events.  However,
focus groups are widgets, too!  Although they cannot be rendered, they
support the same key handler interface as other widgets.  This is how
we create "top-level" key event handlers for the interface.  For
example, if you want to register a handler for a "quit" key such as
\fw{'q'}, the focus group itself is where this key event handler
belongs.  This is because focus groups always try to handle key events
first, and only pass those events onto the focused widget if the focus
group has no matching handler.

\begin{minted}{haskell}
fg <- newFocusGroup
fg `onKeyPressed` \_ key _ ->
  if key == KASCII 'q' then
    exitSuccess else return False
\end{minted}

\subsection{Container Widgets and Input Events}

Although most of the time you will probably end up adding key event
handlers directly to interactive widgets, it may be convenient to wrap
those widgets in containers that affect their behavior.  For example,
in the demonstration in Section \ref{sec:gettingStarted}, we used then
\fw{centered} function to center an edit widget.  The result was a
\fw{Centered} widget, which is one of the many built-in container
widget types.  This type of widget ``relays'' user input events and
focus events to the widget it contains.  This means you can add key
and focus event handlers to the outer widget and they will be passed
on to the child widget for handling.  Most container widgets are
implemented this way; when in doubt about event relaying behavior,
consult the API documentation.

As we saw above, only focused widgets will ever be asked to process
input events; this means that if you add event handlers to a container
such as \fw{Centered}, you'll need to add that widget -- not its child
-- to the focus group that you create.

You might wonder why this is useful.  Consider a situation in which
you want to want to add some paadding to an input widget, such as an
edit widget, but when the edit widget is focused you want to highlight
the padding, too, to make them appear as a single widget.  Since
padding widgets (see Section XXX) relay events to their children, you
could focus the padding widget and the edit widget would automatically
receive the focus as well as user input events.  This kind of focus
and event ``inheritance'' makes it possible to create new, composite
widgets in a flexible way.

\subsection{Merging Focus Groups}

Some widgets, such as the "dialog" widget, are composed of a number of
input widgets already; widgets like "dialog" must create their own
focus groups to provide coherent focus behavior, and they will return
them to you when they are created.  In order to integrate these focus
groups' input widgets into your overall application, you must merge
them with your own focus group.

For example, consider the "directory browser" widget provided by the
library; you might want to place this alongside other widgets that
should also accept input.  When you create the browser widget, you will
get a reference to the widget and a reference to its focus group.

\begin{minted}{haskell}
(browser, fg1) <- newDirBrowser defaultBrowserSkin

fg2 <- newFocusGroup
-- Add my own widgets to fg2

merged <- mergeFocusGroups fg1 fg2
\end{minted}

The \fw{mergeFocusGroups} function will merge the two focus groups and
preserve the order of the widgets, such that widgets in the first
group will come before widgets in the second group in the new group's
focus ordering.  The merged group should then be passed to the rest of
the setup process that we introduced in Section
\ref{sec:gettingStarted}; we'll go into more detail on that in the
next section.
