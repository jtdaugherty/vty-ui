\section{Collections}

The \fw{EventLoop} module provides the \fw{Collection} type, which is
a container for multiple widgets and their \fw{FocusGroup}s with a
pointer to a ``currently-selected'' widget and \fw{FocusGroup}.
\fw{Collection}s are used to construct interfaces as described in
Section \ref{sec:collections}.

To create a new collection:

\begin{haskellcode}
 c <- newCollection
\end{haskellcode}

A \fw{Collection} is not a widget itself, so it cannot be treated like
one.  However, the primary operation of interest is the
\fw{addToCollection} function, which adds an arbitrary \fw{Widget a}
and \fw{FocusGroup} to the \fw{Collection} and returns an \fw{IO}
action which, when run, will switch to that interface and focus group.

\begin{haskellcode}
 switchToFoo <- addToCollection c fooUi fooFocusGroup
 someWidget `onEvent` (const switchToFoo)
\end{haskellcode}

If you choose not to use the \fw{IO} action returned by
\fw{addToCollection}, you may instead call \fw{setCurrentEntry}.  This
function takes a \fw{Collection} and a position and sets the
\fw{Collection}'s current entry to the one at the specified position.
The position is an index into the \fw{Collection}'s internal list
interfaces.  If the position is invalid, a \fw{CollectionError} is
thrown.  This method is not recommended but may be used with care.

\begin{haskellcode}
 _ <- addToCollection c fooUi fooFocusGroup
 someWidget `onEvent` (const $ setCurrentEntry c 0)
\end{haskellcode}

If an empty \fw{Collection} is used in any way, a \fw{CollectionError}
will be thrown.

\subsubsection{Growth Policy}

The growth policy of a \fw{Collection} is always inherited from its
currently-selected interface widget.
